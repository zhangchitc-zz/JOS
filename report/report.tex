\documentclass{article}

\usepackage{csquotes}
% 使用中文CJK包
\usepackage{CJK}
% 图像插入宏包
\usepackage{graphicx}
% 自定义颜色支持
\usepackage[usenames,dvipsnames]{color}
% 长表格跨页支持
\usepackage{longtable}
% 代码高亮支持
\usepackage{listings}
% 算法伪代码包
\usepackage[ruled,vlined]{algorithm2e}
% 自定义标题格式
\usepackage{titlesec}
% 扩展tabular样式
\usepackage{array}
% 添加页眉页脚
\usepackage{fancyhdr}
% 虚拟正文测试
\usepackage{lipsum}
% 数学环境包
\usepackage{amsmath}
% 首行缩进
\usepackage{indentfirst}
% 树状结构图
\usepackage{tree-dvips}
% 脚注环境
\usepackage{footnote}
% 定制表格线
\usepackage{makecell}
% tikz绘图包
\usepackage{tikz}
% URL超链接
\usepackage[dvips, colorlinks, linkcolor=black]{hyperref}
% 断行URL超链接
\usepackage{breakurl}


% hyperref中文兼容
\pdfstringdefDisableCommands{
\let\CJK@XX\relax
\let\CJK@XXX\relax
\let\CJK@XXXp\relax
\let\CJK@XXXX\relax
\let\CJK@XXXXp\relax
}

\usetikzlibrary{positioning,shapes,shadows,arrows}


% 设置脚注在table中可用
\makesavenoteenv{table}

% 设置标题格式
%\titleformat{\chapter}{\raggedright\Huge\bfseries}{Chapter \thechapter}{1em}{}

% 设置默认字体族, 具体字体请查看texdoc psnfss2e

% 设置Roman字体为Palatino
\renewcommand{\rmdefault}{ppl} 
% 设置TypeWriter字体为Courier
\renewcommand{\ttdefault}{pcr} 

% 设置行距
\setlength{\parskip}{1ex}

% 定义需要的颜色

\definecolor{lightgray}{RGB}{230,230,230}
\definecolor{lightblue}{RGB}{224, 224, 255}
\definecolor{darkblue}{RGB}{192, 192, 255}
\definecolor{lightpink}{RGB}{255, 224, 224}
\definecolor{darkpink}{RGB}{255, 192, 192}
\definecolor{keywordyellow}{RGB}{255, 204, 0}
\definecolor{keywordred}{RGB}{194, 58, 0}
\definecolor{numbercolor}{RGB}{102, 51, 0}

% 设置代码风格

% 定义C语言代码风格
\lstdefinestyle{ccode}
{ 
    language=C, 
    numbers=left, 
    numberstyle=\color{numbercolor},
    basicstyle=\scriptsize\ttfamily\bfseries,
    keywordstyle=\color{blue}, 
    commentstyle=\color{PineGreen},
    stringstyle=\color{red}, 
    frame=shadowbox, 
    frameround=tttt,
    breaklines=true,
    backgroundcolor=\color{lightgray} }

% 定义汇编语言代码风格
\lstdefinestyle{acode}
{ 
    language=,
    morekeywords=[1]{mov, movl, movb, movw, orl, xorw, cli, cld, inb, testb, test, jnz, push, pop, jmp, call, lea, add, sub, ret, jle, outb, ljmp, lgdt, cmp, jne, popal, int},
    morekeywords=[2]{ax, bx, cx, dx, eax, ebx, ecx, edx, cr0, cr1, cr2, cr3, al, ds, es, ss, esp, ebp, esi, edi}, 
    morekeywords=[3]{data, text, bss},
    morekeywords=[4]{long, align, p2align, ascii, fill, globl, space, set, rept, byte, word},
    morecomment=[l]\#,
    numbers=left, 
    numberstyle=\color{numbercolor},
    basicstyle=\scriptsize\ttfamily\bfseries,
    keywordstyle=[1]\color{blue}, 
    keywordstyle=[2]\color{keywordyellow},
    keywordstyle=[3]\color{orange},
    keywordstyle=[4]\color{keywordred},
    commentstyle=\color{PineGreen},
    stringstyle=\color{red}, 
    frame=shadowbox, 
    frameround=tttt,
    breaklines=true,
    backgroundcolor=\color{lightgray} }

    
% 定义命令行输出风格
\lstdefinestyle{console}
{
    language=bash, 
    numbers=none, 
    frame=tRBl,
    basicstyle=\scriptsize\color{green}\ttfamily\bfseries,     
    backgroundcolor=\color{black}}


% 定义exercise输出风格
\lstdefinestyle{exercise}
{
    numbers=none, 
    frame=tRBl,
    breaklines=true,
    breakindent=0pt,
    framexleftmargin=1em,
    framexrightmargin=1em,
    framextopmargin=2ex,
    framexbottommargin=2ex,
    xleftmargin=0.05\linewidth,
    xrightmargin=0.05\linewidth,
    basicstyle=\scriptsize\ttfamily\mdseries,   
    moredelim=[is][\ttfamily\bfseries]{|}{|},
    framerule=0.8pt,
    rulecolor=\color{darkblue}, 
    backgroundcolor=\color{lightblue}}
    

% 定义challenge输出风格
\lstdefinestyle{challenge}
{
    numbers=none, 
    frame=tRBl,
    breaklines=true,
    breakindent=0pt,
    framexleftmargin=1em,
    framexrightmargin=1em,
    framextopmargin=2ex,
    framexbottommargin=2ex,
    xleftmargin=0.05\linewidth,
    xrightmargin=0.05\linewidth,
    basicstyle=\scriptsize\ttfamily\mdseries,   
    moredelim=[is][\ttfamily\bfseries]{|}{|},
    framerule=0.8pt,
    rulecolor=\color{darkpink}, 
    backgroundcolor=\color{lightpink}}
    


% 非常重要, listings关闭非ASCII字符兼容
\lstset{extendedchars=false}


% 定义问题的答案格式
\newcommand{\highlight}[1]{{\bfseries \color{red}  #1}}
% 定义函数名格式
\newcommand{\funcname}[1]{{\ttfamily \small #1}}




\pagestyle{fancy}
\begin{document}
\begin{CJK*}{UTF8}{gkai}

\lhead{操作系统实习报告}
\rhead{张弛, 00848231}
\title{操作系统JOS实习第三次报告}
\author{张弛 \hspace{1ex} 00848231, \\
        zhangchitc@gmail.com}

\maketitle
% 记得在文档末尾插入\clearpage
\tableofcontents
\newpage

\section{Introduction}

我在实验中主要参考了华中科技大学邵志远老师写的JOS实习指导,在邵老师的主页上\burl{http://grid.hust.edu.cn/zyshao/OSEngineering.htm} 可以找到。但是这次实验的指导远远不如lab1的指导详尽,所以我这里需要补充的内容会很多。

内联汇编请参考邵老师的第二章讲义,对于语法讲解的很详细。


\section{User-level Environment Creation and Cooperative Multitasking}

这个部分的MIT文档讲解的比较详细,细节的串接都比较清楚。结合代码的注释写起来不是很困难。

\subsection{Round-Robin Scheduling}

\begin{lstlisting}[style=exercise]
|Exercise 1.| Implement round-robin scheduling in sched_yield() as described above. Don't forget to modify syscall() to dispatch sys_yield().

Modify kern/init.c to create three (or more!) environments that all run the program user/yield.c. You should see the environments switch back and forth between each other five times before terminating, like this:

...
Hello, I am environment 00001001.
Hello, I am environment 00001002.
Hello, I am environment 00001003.
Back in environment 00001001, iteration 0.
Back in environment 00001002, iteration 0.
Back in environment 00001003, iteration 0.
Back in environment 00001001, iteration 1.
Back in environment 00001002, iteration 1.
Back in environment 00001003, iteration 1.
...
After the yield programs exit, the idle environment should run and invoke the JOS kernel debugger. If any of this does not happen, then fix your code before proceeding.
\end{lstlisting}

\funcname{sched\_yield()} 函数比较简单,直接贴代码了:

\begin{lstlisting}[style=ccode, title={\scriptsize \ttfamily \bfseries kern/sched.c: sched\_yield()}]
void
sched_yield(void)
{
    struct Env *curenvptr = curenv;

    if (curenv == NULL)
        curenvptr = envs;

    int round = 0;
    for (curenvptr ++; round < NENV; round ++, curenvptr ++) {

        if (curenvptr >= envs + NENV) {
            curenvptr = envs + 1;
        }

        if (curenvptr->env_status == ENV_RUNNABLE)
            env_run (curenvptr);
    }

    // Run the special idle environment when nothing else is runnable.
    if (envs[0].env_status == ENV_RUNNABLE)
        env_run(&envs[0]);
    else {
        cprintf("Destroyed all environments - nothing more to do!\n");
        while (1)
            monitor(NULL);
	}
}
\end{lstlisting}

然后修改kern/syscall.c添加相关的分发机制,然后在kern/init.c中系统启动之初创建user\_idle以后再创建user\_yield,这个用户程序的功能就是作五次\funcname{sys\_yield()}的系统调用,并且切换时打印相关的消息:

\begin{lstlisting}[style=ccode, title={\scriptsize \ttfamily \bfseries kern/init.c: i386\_init()}]
	// Should always have an idle process as first one.
	ENV_CREATE(user_idle);
        ENV_CREATE(user_yield);
        ENV_CREATE(user_yield);
        ENV_CREATE(user_yield);
\end{lstlisting}

那么启动JOS后应该打印出下列消息:(注意,因为是使用Round Robin策略切换,所以顺序应该是确定的)


\begin{lstlisting}[style=console]
qemu -hda obj/kern/kernel.img -serial mon:stdio
6828 decimal is 15254 octal!
Hooray! Passed all test cases for stdlib!!
Physical memory: 66556K available, base = 640K, extended = 65532K
check_page_alloc() succeeded!
page_check() succeeded!
check_boot_pgdir() succeeded!
enabled interrupts: 1 2
	     Setup timer interrupts via 8259A
enabled interrupts: 0 1 2
	     unmasked timer interrupt
[00000000] new env 00001000
[00000000] new env 00001001
[00000000] new env 00001002
[00000000] new env 00001003
Hello, I am environment 00001001.
Hello, I am environment 00001002.
Hello, I am environment 00001003.
Back in environment 00001001, iteration 0.
Back in environment 00001002, iteration 0.
Back in environment 00001003, iteration 0.
Back in environment 00001001, iteration 1.
Back in environment 00001002, iteration 1.
Back in environment 00001003, iteration 1.
Back in environment 00001001, iteration 2.
Back in environment 00001002, iteration 2.
Back in environment 00001003, iteration 2.
Back in environment 00001001, iteration 3.
Back in environment 00001002, iteration 3.
Back in environment 00001003, iteration 3.
Back in environment 00001001, iteration 4.
All done in environment 00001001.
[00001001] exiting gracefully
[00001001] free env 00001001
Back in environment 00001002, iteration 4.
All done in environment 00001002.
[00001002] exiting gracefully
[00001002] free env 00001002
Back in environment 00001003, iteration 4.
All done in environment 00001003.
[00001003] exiting gracefully
[00001003] free env 00001003
Welcome to the JOS kernel monitor!
Type 'help' for a list of commands.
\end{lstlisting}

\vspace{2em}
\hrule
\vspace{2em}


\begin{lstlisting}[style=exercise]
|Question|

In your implementation of env_run() you should have called lcr3(). Before and after the call to lcr3(), your code makes references (at least it should) to the variable e, the argument to env_run. Upon loading the %cr3 register, the addressing context used by the MMU is instantly changed. But a virtual address (namely e) has meaning relative to a given address context--the address context specifies the physical address to which the virtual address maps. Why can the pointer e be dereferenced both before and after the addressing switch?
\end{lstlisting}

我们先来回顾一下

\begin{lstlisting}[style=ccode, firstnumber=257, title={\scriptsize \ttfamily \bfseries inc/mmu.h}]
\end{lstlisting}

\begin{lstlisting}[style=console]
\end{lstlisting}

\begin{lstlisting}[style=console]
\end{lstlisting}

\begin{lstlisting}[style=console]
\end{lstlisting}

\begin{lstlisting}[style=console]
\end{lstlisting}



\begin{lstlisting}[style=exercise]
\end{lstlisting}

\begin{lstlisting}[style=exercise]
\end{lstlisting}

\begin{lstlisting}[style=exercise]
\end{lstlisting}

\begin{lstlisting}[style=exercise]
\end{lstlisting}


\begin{lstlisting}[style=exercise]
\end{lstlisting}

\begin{lstlisting}[style=ccode, firstnumber=257, title={\scriptsize \ttfamily \bfseries inc/mmu.h}]
\end{lstlisting}

\begin{lstlisting}[style=ccode, firstnumber=257, title={\scriptsize \ttfamily \bfseries inc/mmu.h}]
\end{lstlisting}

\begin{lstlisting}[style=ccode, firstnumber=257, title={\scriptsize \ttfamily \bfseries inc/mmu.h}]
\end{lstlisting}

\begin{lstlisting}[style=ccode, firstnumber=257, title={\scriptsize \ttfamily \bfseries inc/mmu.h}]
\end{lstlisting}


\subsection{System Calls for Environment Creation}


\section{Copy-on-Write Fork}

\section{Preemptive Multitasking and Inter-Process communication (IPC)}




\begin{lstlisting}[style=console]
\end{lstlisting}

\begin{lstlisting}[style=exercise]
\end{lstlisting}

\begin{lstlisting}[style=ccode, firstnumber=257, title={\scriptsize \ttfamily \bfseries inc/mmu.h}]
\end{lstlisting}


\begin{lstlisting}[style=acode, title={\scriptsize \ttfamily \bfseries kern/trapentry.S}]
\end{lstlisting}


\clearpage

\end{CJK*}
\end{document}
	

e