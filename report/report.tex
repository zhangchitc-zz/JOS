\documentclass{article}

\usepackage{csquotes}
% 使用中文CJK包
\usepackage{CJK}
% 图像插入宏包
\usepackage{graphicx}
% 自定义颜色支持
\usepackage[usenames,dvipsnames]{color}
% 长表格跨页支持
\usepackage{longtable}
% 代码高亮支持
\usepackage{listings}
% 算法伪代码包
\usepackage[ruled,vlined]{algorithm2e}
% 自定义标题格式
\usepackage{titlesec}
% 扩展tabular样式
\usepackage{array}
% 添加页眉页脚
\usepackage{fancyhdr}
% 虚拟正文测试
\usepackage{lipsum}
% 数学环境包
\usepackage{amsmath}
% 首行缩进
\usepackage{indentfirst}
% 树状结构图
\usepackage{tree-dvips}
% 脚注环境
\usepackage{footnote}
% 定制表格线
\usepackage{makecell}
% tikz绘图包
\usepackage{tikz}
% URL超链接
\usepackage[dvips, colorlinks, linkcolor=black]{hyperref}
% 断行URL超链接
\usepackage{breakurl}


% hyperref中文兼容
\pdfstringdefDisableCommands{
\let\CJK@XX\relax
\let\CJK@XXX\relax
\let\CJK@XXXp\relax
\let\CJK@XXXX\relax
\let\CJK@XXXXp\relax
}

\usetikzlibrary{positioning,shapes,shadows,arrows}


% 设置脚注在table中可用
\makesavenoteenv{table}

% 设置标题格式
%\titleformat{\chapter}{\raggedright\Huge\bfseries}{Chapter \thechapter}{1em}{}

% 设置默认字体族, 具体字体请查看texdoc psnfss2e

% 设置Roman字体为Palatino
\renewcommand{\rmdefault}{ppl} 
% 设置TypeWriter字体为Courier
\renewcommand{\ttdefault}{pcr} 

% 设置行距
\setlength{\parskip}{1ex}

% 定义需要的颜色

\definecolor{lightgray}{RGB}{230,230,230}
\definecolor{lightblue}{RGB}{224, 224, 255}
\definecolor{darkblue}{RGB}{192, 192, 255}
\definecolor{lightpink}{RGB}{255, 224, 224}
\definecolor{darkpink}{RGB}{255, 192, 192}
\definecolor{keywordyellow}{RGB}{255, 204, 0}
\definecolor{keywordred}{RGB}{194, 58, 0}
\definecolor{numbercolor}{RGB}{102, 51, 0}

% 设置代码风格

% 定义makefile代码风格
\lstdefinestyle{mcode}
{ 
    language=make, 
    numbers=left, 
    numberstyle=\color{numbercolor},
    basicstyle=\scriptsize\ttfamily\bfseries,
    keywordstyle=\color{blue}, 
    commentstyle=\color{PineGreen},
    stringstyle=\color{red}, 
    frame=shadowbox, 
    frameround=tttt,
    breaklines=true,
    backgroundcolor=\color{lightgray} }
    
% 定义C语言代码风格
\lstdefinestyle{ccode}
{ 
    language=C, 
    numbers=left, 
    numberstyle=\color{numbercolor},
    basicstyle=\scriptsize\ttfamily\bfseries,
    keywordstyle=\color{blue}, 
    commentstyle=\color{PineGreen},
    stringstyle=\color{red}, 
    frame=shadowbox, 
    frameround=tttt,
    breaklines=true,
    backgroundcolor=\color{lightgray} }

% 定义汇编语言代码风格
\lstdefinestyle{acode}
{ 
    language=,
    morekeywords=[1]{mov, movl, movb, movw, orl, xorw, cli, cld, inb, testb, test, jnz, push, pop, jmp, call, lea, add, sub, ret, jle, outb, ljmp, lgdt, cmp, jne, popal, int, jns, pushw, pushal, pushl, popfl, addl, subl},
    morekeywords=[2]{ax, bx, cx, dx, eax, ebx, ecx, edx, cr0, cr1, cr2, cr3, al, ds, es, ss, esp, ebp, esi, edi}, 
    morekeywords=[3]{data, text, bss},
    morekeywords=[4]{long, align, p2align, ascii, fill, globl, space, set, rept, byte, word},
    morecomment=[l]\#,
    numbers=left, 
    numberstyle=\color{numbercolor},
    basicstyle=\scriptsize\ttfamily\bfseries,
    keywordstyle=[1]\color{blue}, 
    keywordstyle=[2]\color{keywordyellow},
    keywordstyle=[3]\color{orange},
    keywordstyle=[4]\color{keywordred},
    commentstyle=\color{PineGreen},
    stringstyle=\color{red}, 
    frame=shadowbox, 
    frameround=tttt,
    breaklines=true,
    backgroundcolor=\color{lightgray} }

    
% 定义命令行输出风格
\lstdefinestyle{console}
{
    language=bash, 
    numbers=none, 
    frame=tRBl,
    basicstyle=\scriptsize\color{green}\ttfamily\bfseries,     
    backgroundcolor=\color{black}}


% 定义exercise输出风格
\lstdefinestyle{exercise}
{
    numbers=none, 
    frame=tRBl,
    breaklines=true,
    breakindent=0pt,
    framexleftmargin=1em,
    framexrightmargin=1em,
    framextopmargin=2ex,
    framexbottommargin=2ex,
    xleftmargin=0.05\linewidth,
    xrightmargin=0.05\linewidth,
    basicstyle=\scriptsize\ttfamily\mdseries,   
    moredelim=[is][\ttfamily\bfseries]{|}{|},
    framerule=0.8pt,
    rulecolor=\color{darkblue}, 
    backgroundcolor=\color{lightblue}}
    

% 定义challenge输出风格
\lstdefinestyle{challenge}
{
    numbers=none, 
    frame=tRBl,
    breaklines=true,
    breakindent=0pt,
    framexleftmargin=1em,
    framexrightmargin=1em,
    framextopmargin=2ex,
    framexbottommargin=2ex,
    xleftmargin=0.05\linewidth,
    xrightmargin=0.05\linewidth,
    basicstyle=\scriptsize\ttfamily\mdseries,   
    moredelim=[is][\ttfamily\bfseries]{|}{|},
    framerule=0.8pt,
    rulecolor=\color{darkpink}, 
    backgroundcolor=\color{lightpink}}
    


% 非常重要, listings关闭非ASCII字符兼容
\lstset{extendedchars=false}


% 定义问题的答案格式
\newcommand{\highlight}[1]{{\bfseries \color{red}  #1}}
% 定义函数名格式
\newcommand{\funcname}[1]{{\ttfamily \small #1}}




\pagestyle{fancy}
\begin{document}
\begin{CJK*}{UTF8}{gkai}

\lhead{操作系统实习报告}
\rhead{张弛, 00848231}
\title{操作系统JOS实习第六次报告}
\author{张弛 \hspace{1ex} 00848231, \\
        zhangchitc@gmail.com}

\maketitle
% 记得在文档末尾插入\clearpage
\tableofcontents
\newpage

\section{Introduction}

此次Lab是所有JOS实验中最恶心的一次,需要阅读的东西超过了以前所有的总和。所以请做好准备。

在真正下手之前,最好的话请完整的将MIT的材料完整的读一遍,对各个名词和部分有个大致的印象。其他要读的材料还有很多,具体的部分我在报告中会着重提到。


纵观全局,这次Lab的最大难点就是在于你需要\highlight{从零开始}写出一个E100网卡的驱动程序。这个驱动程序从Web Server接收IPC调用向网卡发送数据,然后从网卡接收数据发回给Web Server。这和我们以前的实验都不一样,以前都是给出了结构的框架,我们只需要针对一个具体的功能函数进行细节的填补即可,相关的数据结构、接口设置都为我们设计好了。这次需要我们从头到尾完成整个网卡驱动,困难可想而知。

除开网卡之外的部分都相对简单。因此我们这篇报告重点介绍如何完成这个网卡驱动。操纵网卡我们需要了解的方面有:

\begin{enumerate}
\item{一个PCI设备在JOS中的设置和相关数据结构}
\item{扫描和初始化网卡}
\item{网卡的关键结构}
\item{网卡如何和操作系统交互数据}
\item{如何对网卡发送指令}
\end{enumerate}

报告在后面会一步一步从课程给出的资料中抽取出这些细节。同时我在写的过程中参考了\burl{http://code.google.com/p/os-xv6-network}项目主页提供的一份源代码,该代码对我的帮助非常的大,其中对于PCI设备以及网卡的操纵值得参考。


\section{Initialization and transmitting packets}

对于Network Server的架构,我们只需要大致了解模块即可,这次的实验很少需要对Server进行大规模的修改。

\subsubsection{The Timer Environment}

\begin{lstlisting}[style=exercise]
|Exercise 1|. Add a call to time_tick for every clock interrupt in kern/trap.c. Implement sys_time_msec and add it to syscall in kern/syscall.c so that user space has access to the time.
\end{lstlisting}

这个Exercise太简单了就不贴代码了,唯一需要注意的是在测试用户程序testtime之前由于我们还没有实现网络服务器的部分,所以需要注释掉JOS载入网络服务器的部分:

\begin{lstlisting}[style=ccode, title={\scriptsize \ttfamily \bfseries kern/init.c: i386\_init()}]
	// Should always have an idle process as first one.
	ENV_CREATE(user_idle);

	// Start fs.
	ENV_CREATE(fs_fs);

#if !defined(TEST_NO_NS)
	// Start ns.
	//ENV_CREATE(net_ns);
#endif
\end{lstlisting}

这样运行客户程序才不会出错。


\subsubsection{The Output Environment}

\subsubsection{The Input Environment}

这两部分在材料中也提到了我们需要先实现驱动程序和系统调用部分才可以完成。所以我们先放下他们关注最重要的驱动部分。

\subsection{The Network Interface Card}

\subsubsection{PCI Interface}

\subsubsection{E100 Reset}

\subsubsection{E100 Structure}


\subsubsection{DMA Rings}

\subsection{Device Driver Organization}

\subsection{Transmitting Packets}

\subsubsection{C Structures}

\subsection{Transmitting Packets: Network Server}


\section{Receiving packets and the web server}

\subsection{Receiving Packets}

\subsection{Receiving Packets: Network Server}

\subsection{The Web Server}

\begin{lstlisting}[style=ccode, title={\scriptsize \ttfamily \bfseries : ()}]
\end{lstlisting}

\begin{lstlisting}[style=ccode, title={\scriptsize \ttfamily \bfseries : ()}]
\end{lstlisting}

\begin{lstlisting}[style=ccode, title={\scriptsize \ttfamily \bfseries : ()}]
\end{lstlisting}

\begin{lstlisting}[style=exercise]
\end{lstlisting}

\clearpage

\end{CJK*}
\end{document}
	

e