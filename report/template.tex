\documentclass{article}

% 使用中文CJK包
\usepackage{CJK}
% 图像插入宏包
\usepackage{graphicx}
% 自定义颜色支持
\usepackage[usenames,dvipsnames]{color}
% 长表格跨页支持
\usepackage{longtable}
% 代码高亮支持
\usepackage{listings}
% 算法伪代码包
\usepackage[ruled,vlined]{algorithm2e}
% 自定义标题格式
\usepackage{titlesec}
% 扩展tabular样式
\usepackage{array}
% 添加页眉页脚
\usepackage{fancyhdr}
% 虚拟正文测试
\usepackage{lipsum}
% 首行缩进
\usepackage{indentfirst}
% 树状结构图
\usepackage{tree-dvips}
% 脚注环境
\usepackage{footnote}
% 定制表格线
\usepackage{makecell}
% tikz绘图包
\usepackage{tikz}
% URL超链接
\usepackage[dvips, colorlinks, linkcolor=black]{hyperref}

% hyperref中文兼容
\pdfstringdefDisableCommands{
\let\CJK@XX\relax
\let\CJK@XXX\relax
\let\CJK@XXXp\relax
\let\CJK@XXXX\relax
\let\CJK@XXXXp\relax
}

\usetikzlibrary{positioning,shapes,shadows,arrows}


% 设置脚注在table中可用
\makesavenoteenv{table}

% 设置标题格式
%\titleformat{\chapter}{\raggedright\Huge\bfseries}{Chapter \thechapter}{1em}{}

% 设置默认字体族, 具体字体请查看texdoc psnfss2e

% 设置Roman字体为Palatino
\renewcommand{\rmdefault}{ppl} 
% 设置TypeWriter字体为Courier
\renewcommand{\ttdefault}{pcr} 

% 设置行距
\setlength{\parskip}{1ex}

% 定义需要的颜色

\definecolor{lightgray}{RGB}{230,230,230}
\definecolor{lightblue}{RGB}{224, 224, 255}
\definecolor{darkblue}{RGB}{192, 192, 255}
\definecolor{lightpink}{RGB}{255, 224, 224}
\definecolor{keywordyellow}{RGB}{255, 204, 0}

% 设置代码风格

% 定义C语言代码风格
\lstdefinestyle{ccode}
{ 
    language=C, 
    numbers=left, 
    numberstyle=\color{red},
    basicstyle=\footnotesize\ttfamily\bfseries,
    keywordstyle=\color{blue}, 
    commentstyle=\color{PineGreen},
    stringstyle=\color{red}, 
    frame=shadowbox, 
    frameround=tttt,
    breaklines=true,
    backgroundcolor=\color{lightgray} }
    
    
% 定义汇编语言代码风格
\lstdefinestyle{acode}
{ 
    language=,
    morekeywords=[1]{mov, movl, movb, orl, lgdt},
    morekeywords=[2]{ax, bx, cx, dx, eax, ebx, ecx, edx, cr0, cr1, cr2, cr3}, 
    morekeywords=[3]{data, text, globl, bss},
    morecomment=[l]\#,
    numbers=left, 
    numberstyle=\color{red},
    basicstyle=\footnotesize\ttfamily\bfseries,
    keywordstyle=[1]\color{blue}, 
    keywordstyle=[2]\color{keywordyellow},
    keywordstyle=[3]\color{orange},
    commentstyle=\color{PineGreen},
    stringstyle=\color{red}, 
    frame=shadowbox, 
    frameround=tttt,
    breaklines=true,
    backgroundcolor=\color{lightgray} }
    
    
    
% 定义命令行输出风格
\lstdefinestyle{console}
{
    language=bash, 
    numbers=none, 
    frame=tRBl,
    basicstyle=\footnotesize\color{green}\ttfamily\bfseries,     
    backgroundcolor=\color{black}}


% 定义exercise输出风格
\lstdefinestyle{exercise}
{
    numbers=none, 
    frame=tRBl,
    breaklines=true,
    breakindent=0pt,
    framexleftmargin=1em,
    framexrightmargin=1em,
    framextopmargin=2ex,
    framexbottommargin=2ex,
    xleftmargin=0.05\linewidth,
    xrightmargin=0.05\linewidth,
    basicstyle=\scriptsize\ttfamily\mdseries,   
    moredelim=[is][\ttfamily\bfseries]{|}{|},
    framerule=0.8pt,
    rulecolor=\color{darkblue}, 
    backgroundcolor=\color{lightblue}}
    

% 定义challenge输出风格
\lstdefinestyle{challenge}
{
    numbers=none, 
    frame=tRBl,
    breaklines=true,
    breakindent=0pt,
    framexleftmargin=1em,
    framexrightmargin=1em,
    framextopmargin=2ex,
    framexbottommargin=2ex,
    xleftmargin=0.05\linewidth,
    xrightmargin=0.05\linewidth,
    basicstyle=\scriptsize\ttfamily\mdseries,   
    moredelim=[is][\ttfamily\bfseries]{|}{|},
    framerule=0.8pt,
    rulecolor=\color{darkpink}, 
    backgroundcolor=\color{lightpink}}
    


% 非常重要, listings关闭非ASCII字符兼容
\lstset{extendedchars=false}

\pagestyle{fancy}
\begin{document}
\begin{CJK*}{UTF8}{gkai}

\lhead{操作系统实习报告}
\rhead{张弛, 00848231}
\title{操作系统JOS实习第一次报告}
\author{张弛 \hspace{1ex} 00848231, \\
        zhangchitc@gmail.com}

\maketitle
% 记得在文档末尾插入\clearpage
\tableofcontents
\newpage

\section{fdsa}
fdasfdsa


\section{Template}

C Code:
\begin{lstlisting}[style=ccode]
for i:=maxint to 0 do
begin
{ do nothing }
end;
\end{lstlisting}

Assembly Code:
\begin{lstlisting}[style=acode]
  # Switch from real to protected mode, using a bootstrap GDT
  # and segment translation that makes virtual addresses
  # identical to their physical addresses, so that the
  # effective memory map does not change during the switch.
  
.data
  lgdt    gdtdesc
  movl    %cr0, %eax
  orl     $CR0_PE_ON, %eax
  movl    %eax, %cr0

\end{lstlisting}

Console:
\begin{lstlisting}[style=console]
zhangchi@zhangchi-laptop:~/Hw/lab1$ ls
listingcode.pdf  report.aux  report.out  report.tex
listingcode.tex  report.log  report.pdf  report.toc
[4]+  Done                    gedit listingcode.tex
zhangchi@zhangchi-laptop:~/Hw/lab1$ 

\end{lstlisting}

Exercise:
\begin{lstlisting}[style=exercise]
|Exercise 1.| Familiarize yourself with the assembly language materials available on the 6.828 reference page. You don't have to read them now, but you'll almost certainly want to refer to some of this material when reading and writing x86 assembly.

We do recommend reading the section "The Syntax" in Brennan's Guide to Inline Assembly. It gives a good (and quite brief) description of the AT&T assembly syntax we'll be using with the GNU assembler in JOS.
\end{lstlisting}

Challenge:
\begin{lstlisting}[style=challenge]
|Challenge| Enhance the console to allow text to be printed in different colors. The traditional way to do this is to make it interpret ANSI escape sequences embedded in the text strings printed to the console, but you may use any mechanism you like. There is plenty of information on the 6.828 reference page and elsewhere on the web on programming the VGA display hardware. If you're feeling really adventurous, you could try switching the VGA hardware into a graphics mode and making the console draw text onto the graphical frame buffer.
\end{lstlisting}
\clearpage

\end{CJK*}
\end{document}
	
